\section{Diskussion}
\label{sec:Diskussion}
Die Abweichungen der Leerlaufverstärkung der drei invertierenden Linearverstärker belaufen sich auf 
\begin{align*}
    \frac{V_{1\text{,t}}-V_1}{V_{1\text{,t}}}=&7.1\%\\
    \frac{V_{2\text{,t}}-V_2}{V_{2\text{,t}}}=&9.7\%\\
    \frac{V_{3\text{,t}}-V_1}{V_{3\text{,t}}}=&43.7\%\; .
\end{align*}
Bei den ersten beiden liegen die Messwerte innerhalb der Messunsicherheit und /-genauigkeit. Beim relativen Fehler des dritten Linearverstärkers ist jedoch eine außerordentliche Abweichung zu erkennen.
Grund hierfür könnten neben den oben genannten Fehlern bei der Messung auch systematische Fehler sein, da viele verschiedene Geräte verwendet wurden dessen Intaktheit nicht garantiert ist und die Schaltungen teils sehr aufwändig 
auf dem steckbrett platziert werden mussten.
\\
Beim Integrator und Differentiator weichen die aus den Messwerten berechneten Parameter von den Theoriewerten ab, sie liegen aber dennoch 
in der richtigen Größenordnung. Beim Integrator wird in \autoref{fig:intplot}, die zu erwartende mit der Frequenz abfallende Steigung deutlich. 
Beim Differentiator wird in \autoref{fig:diffplot} deutlich, dass die Verstärkung mit zunehmender Frequenz linear 
abnimmt. Dieses Verhalten bestätigt die theoretische Erwartung.
\\
Es war bei beiden Methoden jeweils nur ein Schwellenwert sichtbar, sodass ein mögliches Mitteln ausblieb. Dadurch wird 
der Schwellenwert ungenauer. Über die beiden Methoden ergaben sich leicht unterschiedliche Schwellwerte, welche dann gemittelt wurden. Der Theoriewert weicht von dem Mittelwert der gemessenen Schwellenwerte $U_+ = 3.104\,\unit{\volt}$ 
lediglich um $3.4\%$ ab.
\\
Der Signalgenerator erzeugt eine Dreiecksspannung, wie in \autoref{fig:genDrei} zu sehen ist. Die Dreiecksspannung hatte in unserem Versuch, 
keine perfekte Form eines Dreiecks. Dies lag a verschiedenen technischen Schwierigkeiten mit dem Oszilloskop. Außerdem zeigen die theoretischen Werte 
von den gemessenen Werten eine sehr große Abweichung. Bei der Frequenz sind dies circa $2500\%$, bei der Amplitude circa $200\%$. Dies könnte ebenfalls 
an den Schwierigkeiten mit dem Oszilloskop, aber auch an der undeutlichen Dreieckspannung liegen.
\\
Beim Generator mit variierenden Amplituden wird in \autoref{fig:genvarAmp} der exponentiell abfallende Verlauf der gedämpften Oszillation 
deutlich. Dies ist als Lösung von der Differentialgleichung (\autoref{eqn:dgllsg}) zu erwarten.