\section{Diskussion}
\label{sec:Diskussion}


Beim Integrator und Differentiator weichen die aus den Messwerten berechneten Parameter deutlich von den Theoriwerten ab, sie liegen aber dennoch 
in der richtigen Größenordnung. Beim Integrator wird in \autoref{fig:intplot}, die zu erwartende mit der Frequenz abfallende Steigung deutlich. 
Die zu Beginn konstante Verstärkung bis hin zu einer Grenzfrequenz wird hingegen weniger deutlich. Vermutlich konnte aus technischen Gründen nicht 
bei einer Frequenz, die klein genug ist gestartet werden. Ein einziger Wert zu Beginn weicht von der linearen Verteilung ab und deutet somit ein 
wenig auf die konstante Verstärkung hin. Beim Differentiator wird in \autoref{fig:diffplot} deutlich, dass die Verstärkung mit zunehmender Frequenz linear 
abnimmt. Dieses Verhalten bestätigt die theoretische Erwartung.


Beim Generator mit variierenden Amplituden wird in \autoref{fig:genvarAmp} der exponentiell abfallende Verlauf der gedämpften Oszillation 
deutlich. Dies ist als Lösung von der Differentialgleichung (\autoref{eqn:dgllsg}) zu erwarten.